\documentclass[letterpaper,10pt]{article}

\usepackage[margin=1in]{geometry}

\usepackage{amsthm,amssymb,amsmath}
%% https://tex.stackexchange.com/a/396829/32270
\allowdisplaybreaks
%% H/T: https://tex.stackexchange.com/a/71153/32270
\usepackage[nottoc,notlot,notlof]{tocbibind}

\usepackage[usenames, dvipsnames]{color}
\usepackage{hyperref}
\hypersetup{
  colorlinks=true,
  urlcolor=blue,
  linkcolor=MidnightBlue,
  citecolor=ForestGreen,
  pdfinfo={
    CreationDate={D:20180822233421},
    ModDate={D:20180822233421},
  },
}

\usepackage{embedfile}
\embedfile{\jobname.tex}

\usepackage{fancyhdr}
\pagestyle{fancy}
\lhead{High-order Solution Transfer between Curved Meshes}
\rhead{Danny Hermes}

\renewcommand{\headrulewidth}{0pt}
\renewcommand{\qed}{\(\blacksquare\)}

\begin{document}

\begin{abstract}
\noindent The problem of solution transfer between meshes arises frequently in
computational physics, e.g. in Lagrangian methods where remeshing
occurs. The interpolation process must be conservative, i.e. it
must conserve physical properties, such as mass. We extend previous
works --- which described the solution transfer process for straight sided
unstructured meshes --- by considering high-order isoparametric meshes
with curved elements. The implementation is highly reliant on accurate
computational geometry routines for evaluating points on and
intersecting B\'{e}zier curves and triangles.
\\ \\
\noindent \emph{Keywords}: Remapping, Curved Meshes, Lagrangian,
Solution Transfer, Numerical analysis
\end{abstract}

\tableofcontents

\section{Introduction}

The first part is a general-purpose tool for computational physics problems.
The tool enables solution transfer across two curved meshes.
Since the tool requires a significant amount of computational geometry, the
second half focuses on computational geometry. In particular, it considers
cases where the geometric methods used have seriously degraded accuracy due to
ill-conditioning.

In computational physics, the problem of solution transfer between meshes
occurs in several applications. For example, by allowing the underyling
computational domain to change during a simulation, computational
effort can be focused dynamically to resolve sensitive features
of a numerical solution. Mesh adaptivity (see, for example,
\cite{Babuska1978, Peraire1987, Pain2001}), this in-flight change in the mesh,
requires translating the numerical solution from the old mesh to the new,
i.e. solution transfer. As another example, Lagrangian or particle-based
methods treat each node in the mesh as a particle and so with each timestep the
mesh travels \emph{with} the fluid (see, for example, \cite{Hirt1974}).
However, over (typically limited) time the mesh
becomes distorted and suffers a loss in element quality which causes
catastrophic loss in the accuracy of computation. To overcome this, the
domain must be remeshed or rezoned and the solution must be
transferred (remapped) onto the new mesh configuration.

When pointwise interpolation is used to transfer a solution, quantities with
physical meaning (e.g. mass, concentration, energy) may not be conserved.
To address this, there have been many explorations (for example,
\cite{Jiao2004, Farrell2009, Farrell2011}) of
\emph{conservative interpolation} (typically using Galerkin or
\(L_2\)-minimizing methods). In this work, the author introduces a
conservative interpolation method for solution transfer between high-order
meshes. These high-order meshes are typically curved, but not necessarily
all elements or at all timesteps.

The existing work on solution transfer has considered straight sided meshes,
which use shape functions that have degree \(p = 1\) to represent solutions
on each element or so-called superparametric elements (i.e. a linear mesh
with degree \(p > 1\) shape functions on a regular grid of points).
However, both to allow for greater geometric flexibility
and for high order of convergence, this work will consider the case
of curved isoparametric\footnote{I.e. the degree of the discrete field on the
mesh is same as the degree of the shape functions that determine the
mesh.} meshes. Allowing curved geometries is useful since many practical
problems involve geometries that change over time, such as flapping flight
or fluid-structure interactions. In addition, high-order CFD methods
(\cite{Wang2013}) have the ability to produce highly accurate solutions
with low dissipation and low dispersion error.

\subsection{Overview}

This work is organized as follows. Section~\ref{sec:preliminaries}
establishes common notation and reviews basic results relevant to the
topics at hand. Section~\ref{sec:bezier-intersection} is an
in-depth discussion of the computational geometry methods needed
to implement to enable solution transfer. Section~\ref{sec:solution-transfer}
describes the solution transfer process and gives results of some
numerical experiments confirming the rate of convergence.

\section{Preliminaries}\label{sec:preliminaries}

Placeholder.

\section{B\'{e}zier Intersection Problems}\label{sec:bezier-intersection}

Placeholder.

\section{Solution Transfer}\label{sec:solution-transfer}

Placeholder.

\bibliography{paper}
\bibliographystyle{alpha}

\end{document}
