\documentclass[letterpaper,10pt]{article}

\usepackage[margin=1in]{geometry}

\usepackage{amsthm,amssymb,amsmath}
%% https://tex.stackexchange.com/a/396829/32270
\allowdisplaybreaks
%% H/T: https://tex.stackexchange.com/a/71153/32270
\usepackage[nottoc,notlot,notlof]{tocbibind}

\usepackage[usenames, dvipsnames]{color}
\usepackage{hyperref}
\hypersetup{
  colorlinks=true,
  urlcolor=blue,
  linkcolor=MidnightBlue,
  citecolor=ForestGreen,
  pdfinfo={
    CreationDate={D:20180822233421},
    ModDate={D:20180822233421},
  },
}

\usepackage{embedfile}
\embedfile{\jobname.tex}

\usepackage{fancyhdr}
\pagestyle{fancy}
\lhead{High-order Solution Transfer between Curved Meshes}
\rhead{Danny Hermes}

\renewcommand{\headrulewidth}{0pt}
\renewcommand{\qed}{\(\blacksquare\)}

\begin{document}

\begin{abstract}
\noindent The problem of solution transfer between meshes arises frequently in
computational physics, e.g. in Lagrangian methods where remeshing
occurs. The interpolation process must be conservative, i.e. it
must conserve physical properties, such as mass. We extend previous
works --- which described the solution transfer process for straight sided
unstructured meshes --- by considering high-order isoparametric meshes
with curved elements. The implementation is highly reliant on accurate
computational geometry routines for evaluating points on and
intersecting B\'{e}zier curves and triangles.
\\ \\
\noindent \emph{Keywords}: Remapping, Curved Meshes, Lagrangian,
Solution Transfer, Numerical analysis
\end{abstract}

\tableofcontents

\section{Introduction}

I will definitely cite this: \cite{Jiao2004}.

\bibliography{paper}
\bibliographystyle{alpha}

\end{document}
