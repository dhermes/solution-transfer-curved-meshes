\documentclass[letterpaper,10pt]{article}

\usepackage[margin=1in]{geometry}

\usepackage{amsthm,amssymb,amsmath,graphicx}
\usepackage{bm}
%% https://tex.stackexchange.com/a/396829/32270
\allowdisplaybreaks
%% H/T: https://tex.stackexchange.com/a/71153/32270
\usepackage[nottoc,notlot,notlof]{tocbibind}

\usepackage[usenames, dvipsnames]{color}
\usepackage{hyperref}
\hypersetup{
  colorlinks=true,
  urlcolor=blue,
  linkcolor=MidnightBlue,
  citecolor=ForestGreen,
  pdfinfo={
    CreationDate={D:20180822233421},
    ModDate={D:20180822233421},
  },
}

\usepackage{embedfile}
\embedfile{\jobname.tex}

\usepackage{fancyhdr}
\pagestyle{fancy}
\lhead{High-order Solution Transfer between Curved Meshes}
\rhead{Danny Hermes}

%% H/T: https://tex.stackexchange.com/a/202047/32270
%%      https://tex.stackexchange.com/a/32463/32270
\usepackage[labelfont=bf]{caption}

\renewcommand{\headrulewidth}{0pt}
\renewcommand{\qed}{\(\blacksquare\)}
\newcommand{\reals}{\mathbf{R}}
\newcommand{\utri}{\mathcal{U}}

\begin{document}

\begin{abstract}
\noindent The problem of solution transfer between meshes arises frequently in
computational physics, e.g. in Lagrangian methods where remeshing
occurs. The interpolation process must be conservative, i.e. it
must conserve physical properties, such as mass. We extend previous
works --- which described the solution transfer process for straight sided
unstructured meshes --- by considering high-order isoparametric meshes
with curved elements. The implementation is highly reliant on accurate
computational geometry routines for evaluating points on and
intersecting B\'{e}zier curves and triangles.
\\ \\
\noindent \emph{Keywords}: Remapping, Curved Meshes, Lagrangian,
Solution Transfer, Numerical analysis
\end{abstract}

\tableofcontents

\section{Introduction}

The first part is a general-purpose tool for computational physics problems.
The tool enables solution transfer across two curved meshes.
Since the tool requires a significant amount of computational geometry, the
second half focuses on computational geometry. In particular, it considers
cases where the geometric methods used have seriously degraded accuracy due to
ill-conditioning.

In computational physics, the problem of solution transfer between meshes
occurs in several applications. For example, by allowing the underyling
computational domain to change during a simulation, computational
effort can be focused dynamically to resolve sensitive features
of a numerical solution. Mesh adaptivity (see, for example,
\cite{Babuska1978, Peraire1987, Pain2001}), this in-flight change in the mesh,
requires translating the numerical solution from the old mesh to the new,
i.e. solution transfer. As another example, Lagrangian or particle-based
methods treat each node in the mesh as a particle and so with each timestep the
mesh travels \emph{with} the fluid (see, for example, \cite{Hirt1974}).
However, over (typically limited) time the mesh
becomes distorted and suffers a loss in element quality which causes
catastrophic loss in the accuracy of computation. To overcome this, the
domain must be remeshed or rezoned and the solution must be
transferred (remapped) onto the new mesh configuration.

When pointwise interpolation is used to transfer a solution, quantities with
physical meaning (e.g. mass, concentration, energy) may not be conserved.
To address this, there have been many explorations (for example,
\cite{Jiao2004, Farrell2009, Farrell2011}) of
\emph{conservative interpolation} (typically using Galerkin or
\(L_2\)-minimizing methods). In this work, the author introduces a
conservative interpolation method for solution transfer between high-order
meshes. These high-order meshes are typically curved, but not necessarily
all elements or at all timesteps.

The existing work on solution transfer has considered straight sided meshes,
which use shape functions that have degree \(p = 1\) to represent solutions
on each element or so-called superparametric elements (i.e. a linear mesh
with degree \(p > 1\) shape functions on a regular grid of points).
However, both to allow for greater geometric flexibility
and for high order of convergence, this work will consider the case
of curved isoparametric\footnote{I.e. the degree of the discrete field on the
mesh is same as the degree of the shape functions that determine the
mesh.} meshes. Allowing curved geometries is useful since many practical
problems involve geometries that change over time, such as flapping flight
or fluid-structure interactions. In addition, high-order CFD methods
(\cite{Wang2013}) have the ability to produce highly accurate solutions
with low dissipation and low dispersion error.

\subsection{Overview}

This work is organized as follows. Section~\ref{sec:preliminaries}
establishes common notation and reviews basic results relevant to the
topics at hand. Section~\ref{sec:bezier-intersection} is an
in-depth discussion of the computational geometry methods needed
to implement to enable solution transfer. Section~\ref{sec:solution-transfer}
describes the solution transfer process and gives results of some
numerical experiments confirming the rate of convergence.

\section{Preliminaries}\label{sec:preliminaries}

\subsection{General Notation}

We'll refer to \(\reals\) for the reals, \(\utri\) represents
the unit triangle (or unit simplex) in \(\reals^2\):
\(\utri = \left\{(s, t) \mid 0 \leq s, t, s + t \leq 1\right\}\).
When dealing with sequences with multiple indices, e.g.
\(s_{m, n} = m + n\), we'll use bold symbols to represent
a multi-index: \(\bm{i} = (m, n)\). We'll use \(\left|\bm{i}\right|\) to
represent the sum of the components in a multi-index.
The binomial coefficient
\(\binom{n}{k}\) is equal to \(\frac{n!}{k! (n - k)!}\) and the trinomial
coefficient \(\binom{n}{i, j, k}\) is equal to \(\frac{n!}{i! j! k!}\)
(where \(i + j + k = n\)). The notation \(\delta_{ij}\) represents the
Kronecker delta, a value which is \(1\) when \(i = j\) and \(0\)
otherwise.

\subsection{B\'{e}zier Curves}

A \emph{B\'{e}zier curve} is a mapping from the unit interval
that is determined by a set of control points
\(\left\{\bm{p}_j\right\}_{j = 0}^n \subset \reals^d\).
For a parameter \(s \in \left[0, 1\right]\), there is a corresponding
point on the curve:
\begin{equation}
b(s) = \sum_{j = 0}^n \binom{n}{j} (1 - s)^{n - j} s^j \bm{p}_j \in
  \reals^d.
\end{equation}
This is a combination of the control points weighted by
each Bernstein basis function
\(B_{j, n}(s) = \binom{n}{j} (1 - s)^{n - j} s^j\).
Due to the binomial expansion
\(1 = (s + (1 - s))^n = \sum_{j = 0}^n B_{j, n}(s)\),
a Bernstein basis function is in
\(\left[0, 1\right]\) when \(s\) is as well. Due to this fact, the
curve must be contained in the convex hull of it's control points.

\subsection{B\'{e}zier Triangles}

A \emph{B\'{e}zier triangle} (\cite[Chapter~17]{Farin2001}) is a
mapping from the unit triangle
\(\utri\) and is determined by a control net
\(\left\{\bm{p}_{i, j, k}\right\}_{i + j + k = n} \subset \reals^d\).
A B\'{e}zier triangle is a particular kind of B\'{e}zier surface, i.e. one
in which there are two cartesian or three barycentric input parameters.
Often the term B\'{e}zier surface is used to refer to a tensor product or
rectangular patch.
For \((s, t) \in \utri\) we can define barycentric weights
\(\lambda_1 = 1 - s - t, \lambda_2 = s, \lambda_3 = t\) so that
\begin{equation}
1 = \left(\lambda_1 + \lambda_2 + \lambda_3\right)^n =
  \sum_{\substack{i + j + k = n \\ i, j, k \geq 0}} \binom{n}{i, j, k}
  \lambda_1^i \lambda_2^j \lambda_3^k.
\end{equation}
Using this we can similarly define a (triangular) Bernstein basis
\begin{equation}
B_{i, j, k}(s, t) = \binom{n}{i, j, k} (1 - s - t)^i s^j t^k
  = \binom{n}{i, j, k} \lambda_1^i \lambda_2^j \lambda_3^k
\end{equation}
that is in \(\left[0, 1\right]\) when \((s, t)\) is in \(\utri\).
Using this, we define points on the B\'{e}zier triangle as a
convex combination of the control net:
\begin{equation}
b(s, t) = \sum_{i + j + k = n} \binom{n}{i, j, k}
  \lambda_1^i \lambda_2^j \lambda_3^k
  \bm{p}_{i, j, k} \in \reals^d.
\end{equation}

\begin{figure}
  \includegraphics{main_figure01.pdf}
  \centering
  \captionsetup{width=.75\linewidth}
  \caption{Cubic B\'{e}zier triangle}
  \label{fig:cubic-bezier-example}
\end{figure}

\noindent Rather than defining a B\'{e}zier triangle by the control net, it can
also be uniquely determined by the image of a standard lattice of
points in \(\utri\): \(b\left(j/n, k/n\right) = \bm{n}_{i, j, k}\);
we'll refer to these as \emph{standard nodes}.
Figure~\ref{fig:cubic-bezier-example} shows these standard nodes for
a cubic triangle in \(\reals^2\). To see the correspondence,
when \(p = 1\) the standard nodes \emph{are} the control net
\begin{equation}
b(s, t) = \lambda_1 \bm{n}_{1, 0, 0} +
\lambda_2 \bm{n}_{0, 1, 0} + \lambda_3 \bm{n}_{0, 0, 1}
\end{equation}
and when \(p = 2\)
\begin{multline}
b(s, t) = \lambda_1\left(2 \lambda_1 - 1\right) \bm{n}_{2, 0, 0} +
\lambda_2\left(2 \lambda_2 - 1\right) \bm{n}_{0, 2, 0} +
\lambda_3\left(2 \lambda_3 - 1\right) \bm{n}_{0, 0, 2} + \\
4 \lambda_1 \lambda_2 \bm{n}_{1, 1, 0} +
4 \lambda_2 \lambda_3 \bm{n}_{0, 1, 1} +
4 \lambda_3 \lambda_1 \bm{n}_{1, 0, 1}.
\end{multline}
However, it's worth noting that the transformation between
the control net and the standard nodes has condition
number that grows exponentially with \(n\) (see \cite{Farouki1991}, which
is related but does not directly show this).
This may make working with
higher degree triangles prohibitively unstable.

A \emph{valid} B\'{e}zier triangle is one which is
diffeomorphic to \(\utri\), i.e. \(b(s, t)\) is bijective and has
an everywhere invertible Jacobian. We must also have the orientation
preserved, i.e. the Jacobian must have positive determinant. For example, in
Figure~\ref{fig:inverted-element}, the image of \(\utri\) under
the map \(b(s, t) = \left[\begin{array}{c c} (1 - s - t)^2 + s^2 & s^2 + t^2
\end{array}\right]^T\) is not valid because the Jacobian is zero along
the curve \(s^2 - st - t^2 - s + t = 0\) (the dashed line). Elements that
are not valid are called \emph{inverted} because they have regions with
``negative area''. For the example, the image \(b\left(\utri\right)\)
leaves the boundary determined by the edge curves: \(b(r, 0)\),
\(b(1 - r, r)\) and \(b(0, 1 - r)\) when \(r \in \left[0, 1\right]\).
This region outside the boundary is traced twice, once with
a positive Jacobian and once with a negative Jacobian.
\begin{figure}
  \includegraphics{inverted_element.pdf}
  \centering
  \captionsetup{width=.75\linewidth}
  \caption{The B\'{e}zier triangle given by \(b(s, t) = \left[
    (1 - s - t)^2 + s^2 \; \; s^2 + t^2 \right]^T\) produces an
    inverted element. It traces the same region twice, once with
    a positive Jacobian (the middle column) and once with a negative
    Jacobian (the right column).}
  \label{fig:inverted-element}
\end{figure}

\subsection{Curved Elements}\label{sec:curved-elements}

We define a curved mesh element \(\mathcal{T}\) of degree \(p\)
to be a B\'{e}zier triangle in \(\reals^2\) of the same degree.
We refer to the component functions of \(b(s, t)\) (the map that
gives \(\mathcal{T} = b\left(\utri\right)\)) as \(x(s, t)\) and \(y(s, t)\).

This fits a typical definition (\cite[Chapter~12]{FEM-ClaesJohnson})
of a curved element, but gives a special meaning to the mapping from
the reference triangle. Interpreting elements as B\'{e}zier triangles
has been used for Lagrangian methods where
mesh adaptivity is needed (e.g. \cite{CardozeMOP04}). Typically curved
elements only have one curved side (\cite{McLeod1972}) since they are used
to resolve geometric features of a boundary. See also
\cite{Zlmal1973, Zlmal1974}.
B\'{e}zier curves and triangles have a number of mathematical properties
(e.g. the convex hull property) that lead to elegant geometric
descriptions and algorithms.

Note that a B\'{e}zier triangle can be
determined from many different sources of data (for example the control net
or the standard nodes). The choice of this data may be changed to suit the
underlying physical problem without changing the actual mapping. Conversely,
the data can be fixed (e.g. as the control net) to avoid costly basis
conversion; once fixed, the equations of motion and other PDE terms can
be recast relative to the new basis (for an example, see \cite{Persson2009},
where the domain varies with time but the problem is reduced to
solving a transformed conservation law in a fixed reference configuration).

\subsection{Shape Functions}\label{subsec:shape-functions}

When defining shape functions (i.e. a basis with geometric meaning) on a
curved element there are (at least) two choices. When the degree of the
shape functions is the same as the degree of the function being
represented on the B\'{e}zier triangle,
we say the element \(\mathcal{T}\) is \emph{isoparametric}.
For the multi-index
\(\bm{i} = (i, j , k)\), we define \(\bm{u}_{\bm{i}} =
\left(j/n, k/n\right)\) and the corresponding standard node
\(\bm{n}_{\bm{i}} = b\left(\bm{u}_{\bm{i}}\right)\).
Given these points, two choices for shape functions present
themselves:
\begin{itemize}
  \itemsep 0em
  \item \emph{Pre-Image Basis}:
    \(\phi_{\bm{j}}\left(\bm{n}_{\bm{i}}\right) =
      \widehat{\phi}_{\bm{j}}\left(\bm{u}_{\bm{i}}\right) =
      \widehat{\phi}_{\bm{j}}\left(b^{-1}\left(
      \bm{n}_{\bm{i}}\right)\right)\)
    where \(\widehat{\phi}_{\bm{j}}\) is a canonical basis function
    on \(\utri\), i.e.
    \(\widehat{\phi}_{\bm{j}}\) a degree \(p\) bivariate polynomial and
    \(\widehat{\phi}_{\bm{j}}\left(\bm{u}_{\bm{i}}\right) =
    \delta_{\bm{i} \bm{j}}\)
  \item \emph{Global Coordinates Basis}:
    \(\phi_{\bm{j}}\left(\bm{n}_{\bm{i}}\right) =
    \delta_{\bm{i} \bm{j}}\), i.e. a canonical basis function
    on the standard nodes \(\left\{\bm{n}_{\bm{i}}\right\}\).
\end{itemize}

\noindent For example, consider a quadratic B\'{e}zier triangle:
\begin{gather}
b(s, t) = \left[ \begin{array}{c c}
    4 (s t + s + t) & 4 (s t + t + 1)
  \end{array}\right]^T \\
\Longrightarrow
\left[ \begin{array}{c c c c c c}
    \bm{n}_{2, 0, 0} &
    \bm{n}_{1, 1, 0} &
    \bm{n}_{0, 2, 0} &
    \bm{n}_{1, 0, 1} &
    \bm{n}_{0, 1, 1} &
    \bm{n}_{0, 0, 2}
  \end{array}\right] = \left[ \begin{array}{c c c c c c}
    0 & 2 & 4 & 2 & 5 & 4 \\
    4 & 4 & 4 & 6 & 7 & 8
  \end{array}\right].
\end{gather}
In the \emph{Global Coordinates Basis}, we have
\begin{equation}
\phi^{G}_{0, 1, 1}(x, y) = \frac{(y - 4) (x - y + 4)}{6}.
\end{equation}
For the \emph{Pre-Image Basis}, we need the inverse
and the canonical basis
\begin{equation}
b^{-1}(x, y) = \left[ \begin{array}{c c}
    \frac{x - y + 4}{4} & \frac{y - 4}{x - y + 8}
  \end{array}\right] \quad \text{and} \quad
\widehat{\phi}_{0, 1, 1}(s, t) = 4 s t
\end{equation}
and together they give
\begin{equation}
\phi^{P}_{0, 1, 1}(x, y) = \frac{(y - 4) (x - y + 4)}{x - y + 8}.
\end{equation}
In general \(\phi_{\bm{j}}^P\) may not even be a rational bivariate
function; due to composition with \(b^{-1}\) we can only guarantee that
it is algebraic (i.e. it can be defined as the zero set of polynomials).

\subsection{Curved Polygons}\label{subsec:curved-polygons}

\begin{figure}
  \includegraphics{main_figure26.pdf}
  \centering
  \captionsetup{width=.75\linewidth}
  \caption{Intersection of B\'{e}zier triangles form a curved polygon.}
  \label{fig:bezier-triangle-intersect}
\end{figure}

When intersecting two curved elements, the resulting surface(s) will
be defined by the boundary, alternating between edges of each
element.
For example, in Figure~\ref{fig:bezier-triangle-intersect}, a
``curved quadrilateral'' is formed when two B\'{e}zier triangles
\(\mathcal{T}_0\) and \(\mathcal{T}_1\) are intersected.

A \emph{curved polygon} is defined by a collection of B\'{e}zier curves
in \(\reals^2\) that determine the boundary. In order to be
a valid polygon, none of the boundary curves may cross, the
ends of consecutive edge curves must meet and the curves must be right-hand
oriented. For our example in
Figure~\ref{fig:bezier-triangle-intersect}, the triangles
have boundaries formed by three B\'{e}zier curves:
\(\partial \mathcal{T}_0 = b_{0, 0} \cup b_{0, 1} \cup b_{0, 2}\) and
\(\partial \mathcal{T}_1 = b_{1, 0} \cup b_{1, 1} \cup b_{1, 2}\).
The intersection \(\mathcal{P}\) is defined by four boundary
curves: \(\partial \mathcal{P} =
C_1 \cup C_2 \cup C_3 \cup C_4\). Each boundary
curve is itself a B\'{e}zier curve\footnote{A specialization of a
B\'{e}zier curve \(b\left(\left[a_1, a_2\right]\right)\)
is also a B\'{e}zier curve.}:
\(C_1 = b_{0, 0}\left(\left[0, 1/8\right]\right)\),
\(C_2 = b_{1, 2}\left(\left[7/8, 1\right]\right)\),
\(C_3 = b_{1, 0}\left(\left[0, 1/7\right]\right)\) and
\(C_4 = b_{0, 2}\left(\left[6/7, 1\right]\right)\).

Though an intersection can be described in terms of the B\'{e}zier triangles,
the structure of the control net will be lost. The region will not in general
be able to be described by a mapping from a simple space like
\(\utri\).

\section{B\'{e}zier Intersection Problems}\label{sec:bezier-intersection}

Placeholder.

\section{Solution Transfer}\label{sec:solution-transfer}

Placeholder.

\bibliography{paper}
\bibliographystyle{alpha}

\end{document}
