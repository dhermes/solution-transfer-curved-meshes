\documentclass{letter}
\usepackage{amssymb,amsmath}
\usepackage{graphicx}
\usepackage{scrextend}
\usepackage{bm}
\usepackage{color}

\oddsidemargin=0in
\evensidemargin=0in
\textwidth=6.5in
\topmargin=-.5in
\textheight=8.5in

\renewcommand{\rq}[1]{
\vspace{8pt}
\begin{addmargin}[2em]{2em}
\it
\color{blue}{    
  #1
}
\end{addmargin}\vspace{12pt}}

\begin{document}
\begin{letter}{}

\opening{}

Commun. Appl. Math. Comput. Sci.\\[10pt]
\textit{Manuscript}: HIGH-ORDER SOLUTION TRANSFER BETWEEN CURVED TRIANGULAR
MESHES \\[10pt]
\textbf{Revisions and Responses to Reviewers}\\[10pt]

We would like to thank all reviewers for the insightful and helpful comments.
These comments and corrections have proven to be very useful in correcting
errors and improving the quality of the manuscript.

Responses to general and specific comments are found below.

\closing{Sincerely, \\ \ \\ \ \\
\parbox{\textwidth}{
Danny Hermes\\ \ \\
Per-Olof Persson\\
Department of Mathematics,\\University of California, Berkeley,\\
Berkeley, CA, 94720-3840, USA\\
\texttt{persson@berkeley.edu}}}

\pagebreak
\clearpage

\textbf{Reviewer C.}

\textbf{General comments.}

\begin{itemize} \renewcommand\labelitemi{--}

\item \rq{
This paper presents a L2 Galerkin projection method for a globally
conservative transfer of solutions between two higher-order curvilinear
triangular meshes. The authors use B\'{e}zier curves to represent the curved
edges of the meshes. Their procedure consists of three steps: \\
\ \\
1. Finding candidate source cells that potentially intersect each target mesh
   This is done by a brute force search for the first pair and then a walking
   algorithm to find candidates for the rest of the target triangles. They
   say their method is \(\mathcal{O}\left(N_d + N_t\right)\) which is better
   than what would be obtained
   by a tree search --- \(\mathcal{O}\left(N_t \log\left(N_d\right)\right)\) for
   sufficiently large meshes depending on
   the constants.\\
\ \\
2. Computing the domains of intersection between the B\'{e}zier triangles
   This procedure detailed is the Appendix A.1 / A.2. First they use a
   subdivision of B\'{e}zier curves to localize intervals in which intersection
   of each pair of lies. They then use Newtwon's method to find the actual
   point of intersection which they point out has trouble for nearly tangential
   cases. No particular fall back solution is offered for cases where Newton's
   method has trouble.\\
\ \\
3. Computing the matrices required to solve for the target field
   This involves integrating of weak forms of the integrals over the
   intersection domains of the donor and target triangles. This is done by
   transforming the area integrals into integrals over the curved edges of the
   intersection domains.\\
 \ \\
The authors present convergence results for three fields with meshes
of different order on a square and circular domain. The results show
faithful reproduction of the expected order of convergence. Of course
the weakness of the method is the computation of the intersection of
domains which is prone to convergence issues. The second weakness is
the difficulty in extending the technique to 3D. \\
\ \\
The paper is definitely one of a handful to tackle the difficult topic
of remapping between higher order meshes. Notably, the authors have
omitted to mention the work of Anderson, Dobrev, Kolev and Rieben on
``Monotonicity oin High-Order Curviliniear Finite Element ALE Remap."
It is certainly an important topic, one that has constrained the use
of High-order elements for ALE simulations.}

  Thank you for this summary, and yes we completely agree that 3D and geometry
  robustness is left for future work. We appreciate the reference to
  monotonic remaps, we have added this and other references to the new revision.

\item \rq{
The paper is generally well written and presents a clear mathematical
description of the methods used. Only the introduction is a little
meandering and at times repeats itself --- this could use some
tightening up.}

  We agree on this, and have made a major rewrite of the entire introduction.
  In particular, we removed Fig 1.1 and shorted it by about 50\%.

\item \rq{
Suggestions for improvement:
1. Incorporate A.1 and A.2 into main text. These are essential parts of the
procedure and there is no good reason to relegate them to the Appendix.}

  We looked into this, but prefer to keep them as separate appendices.
  Although an important part of the procedure, B\'ezier intersections
  by subdivisions is a big topic with lots of prior work, and not our
  main contribution. So we think the main text gets easier to read by
  leaving these details to the end.

\item \rq{
2. Present results of remap against two different meshes of domains in addition
to convergence results for nested refinement. Some metrics could be
conservation error, and L2 norm of the approximation error.}

  Note that our nested refinement convergence study does also include
  the transfer from a different mesh, at each level of refinement, and
  the errors are in the L2 norm. So we felt that this is a fairly
  complete experimental check for correctness. We looked into adding
  conservation errors, but since they are zero analytically by design, we
  only got machine tolerance errors and used it mostly as a verification of
  our codes.
  

\item \rq{
3. Some discussion of bounds preservation or monotonicity of the solution
should be incorporated (important for non-physical solutions)}

  Good suggestion, we have added a discussion about this in the introduction
  with citations. Although beyond the scope of this paper, we believe it is
  likely that our techniques can be extended to other types of projections.

\item \rq{
4. Suggest reasonable fall-back strategies for when the intersection solution
runs into trouble.}

  We added a comment about this at in section A.2. In short, our implementation
  worked fine for all our tests, but in general it is of course true that one
  would need special cases and fallback strategies.

\end{itemize}

\textbf{Reviewer Y.}

\textbf{Summary:}

\begin{itemize} \renewcommand\labelitemi{--}

\item \rq{
The authors develop a methodology for mapping / interpolating data
from a given ``donor" mesh to a given ``target" mesh (with different
topologies / geometries). The proposed method is valid for 2D triangular meshes
of high-polynomial degree and for scalar fields approximated with DG basis
functions of high-polynomial degree. Such a technique is quite useful for
many applications including code-to-code coupling of solution data between
different computational meshes, embedded meshing methods and remapping methods
in arbitrary Lagrangian-Eulerian approaches. The proposed method is
conservative by design and makes use of novel techniques based on
B\'{e}zier-curves and general integration over curved polygons. While limited
in its scope (e.g., 2D, triangular topologies, scalar fields), the proposed
method is both novel and impressive. The paper is well written and clearly
explains all of its concepts / methodologies.}

Thank you very much!
  
\end{itemize}

\textbf{Detailed Comments:}

\begin{itemize} \renewcommand\labelitemi{--}

\item \rq{
If this technique were to be used in ALE codes for the remapping phase, there
must be some additional work done to handle the case of spatially discontinuous
fields. In ALE hydrodynamics, material contacts / interfaces can be captured
exactly during the Lagrangian phase by aligning the computational mesh with
the material discontinuity. During the ALE remap phase, this discontinuity
must be handled without violating local bounds (i.e. introducing new local
extremum / oscillations). For remapping methods which are based on solving a
transport (advection) equation from the old (or donor) to the new (or target)
mesh, this is typically handled using flux-corrected transport methods which
blend monotone 1st order methods with high order methods in a non-linear way.
One possible approach for handling this in the author's proposed method is to
use a type of limited FCT based L2 projection for calculating the overlap
integrals as proposed in [1]. In this technique, element-wise L2 projection
can be done that is guaranteed monotone even for discontinuous functions
provided a well-known monotone low order element-based solution can be found.}

  Thank you for these suggestions, and note that the previous reviewer had
  similar comments as well. We have added a paragraph of discussion of this,
  including references, but left the actual work for future extensions.

\item \rq{
The graphic depiction of figure 1.1 is useful, but it could benefit from having
different colors / markers to designate the particles associated with the
initial element for each of the time slices in the top row. For example, in the
second column, top row of the figure, the points associated with \(t = 0.5\)
should have a different color or marker than the ones associated
with \(t = 0.0\).}

  Thank you, we did at first update the figure to have more clear
  colors, but then based on suggestions from the other reviewers we
  ended up shortening the introduction and removing this figure from
  the paper!

\item \rq{
I believe the brute force search that is used in section 3.3 could be further
improved by using a binning technique which makes use of a ``spatial
acceleration data structure" (see for example [2]).}

Thank you for this helpful note. We have added a sentence to the end of the
last paragraph in section 3.3: ``Spatial binning techniques for mesh elements
could also be used to improve the initial brute-force search as
in [WZRC16]."

\item \rq{
In the numerical experiments, it is not clear from Figure 4.1 that any of the
donor / target meshes had high-degree polynomial curvature for either the
exterior or interior elements --- was this the case? If so, I believe an image
illustrating the curvature of both the donor / target meshes would be useful to
show.}

Thank you for pointing this out. The green circular region on the right has
been updated to reflect the curvature. The image was previously produced with
a ``shortcut" to reduce image size that just used linear approximations (i.e.
connect the corners) of each edge. The updated image uses 32 points (evenly
spaced in parameter space) along each edge to better reflect the curvature
while still keeping the image size small.

\item \rq{
I would be very curious to see if the authors are able to extend this technique
to 3D meshes as well as quadrilateral / hexahedral element topologies of
arbitrary polynomial degree. It would also be interesting to learn if such a
technique could be used for remapping vector valued field in the spaces
\(H^1_{d}\), \(H\left(\operatorname{curl}\right)\) and
\(H\left(\operatorname{div}\right)\). For the
\(H\left(\operatorname{curl}\right)\) and \(H\left(\operatorname{div}\right)\)
case, the notion of conservation imposes additional constraints (e.g.
preservation of divergence / curl) which might require new techniques in this
geometric based remapping approach.}

  Yes we completely agree. The vector fields and additional constrains can
  probably be incorporated into our procedure, since similar work has been
  done before. Quads are probably very similar to triangles. However, 3D
  is very complex for both tets and hexes so that might need fundamental
  new developments.

\end{itemize}

\textbf{Reviewer M.}

\textbf{Comments from PDF:}

\begin{itemize} \renewcommand\labelitemi{--}

\item \rq{Lot of repetition in Introduction}

  We agree, and the first reviewer made the same comment. We have rewritten
  and shortened it significantly.

\item \rq{
How can it ``leave" the domain? You either have boundary
conditions that prevent the nodes from going out of the domain or you
have a free boundary that moves out.}

You're right, we deleted the remark ``...if it causes the mesh to leave the
domain being analyzed...".

\item \rq{Define ALE}

  We added the definition ``Arbitrary Lagrangian-Eulerian'' at the place it
  is first mentioned.
  
\item \rq{Sentence structure needs to be fixed}

The sentence fragment ``Or, if two different methods use two different meshes
of the same domain." has been modified to end with ``domain, a solution may
need to be transferred across these meshes for comparison."

\item \rq{Not always. Some ALE methods don't have any mesh modification}

Done. Added caveat ``Many ALE methods modify the mesh during simulation:
when flow-based mesh distortion occurs..." to the beginning of the sentence
that previously started out ``When flow-based mesh distortion occurs...".

\item \rq{Interpolation \(\to\) interpolative?}

Done.

\item \rq{it's \(\to\) its}

Done.

\item \rq{has an invertible Jacobian everywhere}

Done.

\item \rq{CG}

Changed ``In the CG case, some coefficients will..." to ``In CG (continuous
Galerkin, as opposed to DG or discontinuous Galerkin), some coefficients
will..."

\item \rq{term \(\to\) turn}

Done.

\end{itemize}

\end{letter}

\end{document}
